
  \section{Introduction}



    \subsection{Motivation of Research}
   	\indent \indent The ability to predict the future data, based on past data, makes an important leverage that can push organization forward.
      Time series forecasting is an important tool under this scenario where the goal is to predict the behavior of complex systems solely by looking at patterns in past data. 
       A time series is a collection of periodic ordered observations  that appears  in multiple range of domains such as literature, agriculture, finance, media, etc., just to name a few \cite{paulo}. 
  	 Time series are very complex because each observation is dependent upon the previous observations and often is influenced by more than one previous observations \cite{anne}. Hence, this research mainly focuses on prediction of time series data in web traffic domain.\\
    Information about the potential web traffic that a website might receive in future is essential to the management of server resources and potentially prevent the Denial of Service (DoS) during peak hours. Online advertising business also depends upon predicted web traffic because they need to distribute advertisements to the selected websites based on the predicted web traffic \cite{rojaa}.
    As the estimation of web traffic is important due to aforementioned purposes, several research works have been done in this sector.\\
  
    \subsection{Prior Works}
  
    Traditionally these time series forecasting has been performed using various statistical-based methods \cite{methodsapplications}. The major drawback of most of the statistical models is that, they consider the time series are generated from a linear process. However, most of the real world time series generated often consists of temporal and/or spatial variability and suffered from nonlinearity of underlying data generating process \cite{Panigrahi_timeseries}. 
  In \cite{rojaa}, the author used both MLP and RNN for the prediction of web traffic and found out that MLP outperformed RNN. Despite extensive experimentation which involved tuning various parameters, the predictions produced by RNN were consistently poor in terms of prediction accuracy \cite{rojaa}.
  Work in \cite{chakraborty} used Neural Network to forecast prices of flour in three different cities. They compared their results with autoregressive moving average (ARMA)  model and found that Neural Network approach outperformed the ARMA model. 

  In \cite{winner}, the author uses Recurrent Neural Network sequence to sequence model where he used data features such as days of the week, year to year autocorrelation, quarter to quarter autocorrelation and page popularity to train his model.  His model outperformed all the model in a web traffic forecasting competition organized by Google on Kaggle (a platform for predictive modelling and analytics competitions).
  Another participant of the same competition used Kalman filter to make the prediction \cite{otto}. % For the traffic generated by bots otto discarded weekly seasonality as bots don’t care about weekly seasonality which improved his model performance\cite{otto}.

  All the methods that have been used before have some sort of disadvantages. ANNs require a complex training process, they may converge to a local minimum, they have difficulty in determining the optimum network structure, and they experience fading memory (FM) \cite{wind}. Therefore, it is difficult to create a more accurate web traffic prediction model using ANNs.  Echo state networks (ESNs) are a novel type of ANN proposed by Prof. Jaeger in 2001 and are regarded as the closest representation of the learning process of the human brain \cite{Jaeger:science}. 
  ESNs can effectively solve all of the aforementioned problems with ANNs. In addition, the computational requirement for training ESN is lesser than training ANNs and MLPs because while training ESN the connection weights of the reservoir are not changed and only weights from the reservoir to the output units are adapted, so training becomes a linear regression task \cite{Holzmann07echostate, wind}.  Therefore, we have chosen to use ESNs for the prediction of web traffic.\\

  \subsection{Objective}
  Our main objectives to conduct this research are listed below:
  \begin{itemize}
  	\item To evaluate the performance of ESN for the prediction of web traffic.
  	\item To optimize the parameters of ESN for it's best performance.
    \end{itemize}
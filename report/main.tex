\documentclass[a4paper,11pt,oneside]{article}

% To use this template, you have to have a halfway complete LaTeX
% installation and you have to run pdflatex, followed by bibtex,
% following by one-two more pdflatex runs.
%
% Note thad usimg a spel chequer (e.g. ispell, aspell) is generolz
% a very guud ideo.

\usepackage[a4paper,top=3cm,bottom=3cm,left=3cm,right=3cm]{geometry}
\renewcommand{\familydefault}{\sfdefault}
\usepackage{helvet}
\usepackage{parskip}		%% blank lines between paragraphs, no indent
\usepackage[pdftex]{graphicx}	%% include graphics, preferrably pdf
\usepackage[pdftex]{hyperref}	%% many PDF options can be set here
\pdfadjustspacing=1		%% force LaTeX-like character spacing

\newcommand{\myname}{Lalit Singh}
\newcommand{\mytitle}{Web Traffic Prediction using Echo State Networks}
\newcommand{\mysupervisor}{Prof. Herbert Jaeger}

\hypersetup{
  pdfauthor = {\myname},
  pdftitle = {\mytitle},
  pdfkeywords = {},
  colorlinks = {true},
  linkcolor = {blue}
}


\begin{document}
  \pagenumbering{roman}
    \thispagestyle{empty}  \begin{flushright}    \includegraphics[scale=0.7]{bsc-logo}  \end{flushright}  \vspace{20mm}  \begin{center}    \huge    \textbf{\mytitle}  \end{center}  \vspace*{4mm}  \begin{center}   \Large by  \end{center}  \vspace*{4mm}  \begin{center}    \Large    \textbf{\myname}  \end{center}  \vspace*{20mm}  \begin{center}    \large    Bachelor Thesis in Computer Science  \end{center}  \vfill  \begin{flushright}    \large    \begin{tabular}{l}      \mysupervisor \\      \hline      Bachelor Thesis Supervisor \\      \\    \end{tabular}  \end{flushright}  \vspace*{8mm}  \begin{flushleft}    \large    Date of Submission: \today \\    \rule{\textwidth}{1pt}  \end{flushleft}  \begin{center}    \Large Jacobs University --- Focus Area Mobility  \end{center}
  
    \newpage
  \thispagestyle{empty}

  With my signature, I certify that this thesis has been written by me
  using only the indicates resources and materials. Where I have
  presented data and results, the data and results are complete,
  genuine, and have been obtained by me unless otherwise acknowledged;
  where my results derive from computer programs, these computer
  programs have been written by me unless otherwise acknowledged. I
  further confirm that this thesis has not been submitted, either in
  part or as a whole, for any other academic degree at this or another
  institution.

  \vspace{20mm}

  Signature \hfill Place, Date

 
  
  
  \section{Introduction}

  This, like the rest, addresses fellow experts from your field (but
  not from your particular topic of research). Here you should
  technically connect to the main concepts from that field and give an
  outline of your project, stating the research/engineering question
  that you want to get answered by your project.

  (target size: 1 page)

  \section{Statement and Motivation of Research}

  This part should make clear which question, exactly, you are
  pursuing, and why your project is relevant/interesting. This is the
  place to cite relevant literature. Where does your project extend
  the state of the art? What weaknesses in known approaches do you
  hope to overcome? If you have carried out preliminary experiments,
  describe them here.

  (target size: 3-5 pages)

  \section{Description of the Investigation}

  This is the technical core of the thesis. Here you lay out your how
  you answered your research question, you specify your design of
  experiments or simulations, point out difficulties that you
  encountered, etc.

  (target size: 3-4 pages)

  \section{Evaluation of the Investigation}

  This section discusses criteria that are used to evaluate the
  research results. Make sure your results can be used to published
  research results, i.e., to the already known state-of-the-art.

  (target size: 1-3 page)

  \section{Conclusions}

  Summarize the main aspects and results of the research
  project. Provide an answer to the research questions stated earlier.

  (target size: 1/2 page)

  \nocite{JS06}

  \newpage
  \bibliographystyle{unsrt}
  \bibliography{bsc-sample}

\end{document}
